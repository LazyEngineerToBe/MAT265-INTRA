\section{Application des É.D}
\vspace{-2\baselineskip}
\subsection*{Cas particuliers}

\subsubsection{Chute libre}
\raggedright
Équation du mouvement: \(\frac{dv}{dt}+\frac{k}{m}v=g \;\mid v(0)=v_0\)\\
Avec:\\
$k$: coefficient d'amortissement
\begin{gather*}
 v\text{: vitesse} \qquad m\text{: masse} \qquad g\text{: } 9.81 m/s^2
\end{gather*}
É.D linéaire avec:
\begin{gather*}
  p(x)=\frac{k}{m} \qquad  q(x)=g
\end{gather*}
Vitesse limite: \(v_\infty = \frac{mg}{k}\)\\\vspace{2.5pt}
Vitesse instantanée: \[v(t)=v_0 e^{-\frac{k}{m}t}+\frac{mg}{k}(1-e^{-\frac{k}{m}t})\]

\subsubsection*{Circuit RL}
$V_L = L\frac{di}{dt}$\\\vspace{5pt}
$V_R=RI$\\\vspace{5pt}
Loi de kirchoff: \(V_L+V_R=V\)\\
*On suppose V constant\\\vspace{5pt}
Équation: \(\frac{di}{dt}=\frac{R}{L}i=\frac{V}{L} |i(o)=i_0\)\\
É.D de Bernouilli avec:
\begin{gather*}
 y=i \qquad p(x)=\frac{R}{L} \qquad q(x)=\frac{V}{L}
\end{gather*}
Courant limite: \(i_\infty=\frac{V}{R}\)\\
Courant instantané: \(i(t)=i_0e^{-\frac{R}{L}t}+\frac{V}{R}(1-e^{-\frac{R}{L}t})\)\\
$to=\frac{L}{R}=62.2\%$ = constante de temps\\\vspace{5pt}
$T=5e=99.3\%$ = Temps de réponse\\\vspace{5pt}

\subsubsection*{Circuit RC}
$V_R = RI = R_C\frac{dV_C}{dt}$\\\vspace{5pt}
$V_C=\frac{q}{C} = \frac{1}{C}\frac{dq}{dt}$ avec q=charge (coulombs)\\\vspace{5pt}
$i=\frac{dq}{dt}=C\frac{dV_C}{dt}$\\\vspace{5pt}
Loi de Kirchoff: $V=V_R+V_C=R_C\frac{dV_C}{dt}+V_C$\\\vspace{5pt}
Équation caractéristique: \[\frac{dV_C}{dt}+\frac{1}{R_C}V_C=\frac{V}{R_C}\;\mid V_C(0)=V_{C_{0}}\]
*avec V constant\\
\begin{gather*}
    y=V_C \qquad P(x)=\frac{1}{R_C} \qquad q(x)=\frac{V}{R_C}
\end{gather*}
$V_C$ limite: \(V_{C_{\infty}} = V\)\\
$V_C$ instantanée: \(V_C(t)=V_{C_{0}}e^{-\frac{t}{R_C}}+V(1-e^{-\frac{t}{R_C}})\)\\
On suppose que le condensateur n'était pas chargé $V_C(0)=0$\\
Tension instantannée: \(V_C(t)=V(1-e^{-\frac{t}{R_C}})\)\\
$to=RC = 63.21\%$: constante de temps\\
$T=5RC=99.3\%$: temps de réponse

\subsubsection*{Loi de refroidissement de Newton}
$T$ = température de l'objet\\
$M$ = température du milieu\\
$k$ = coefficient de proportionnalité\\
\[\frac{dT}{dt}=-k(T-M) \:\mid  k>0\]
Équation type: \[\frac{dT}{dt}+kT=kM \;\mid T(0)=t_0\]
\begin{gather*}
 y=T \qquad   p(x)=k \qquad q(t)=kM
\end{gather*}
Température limite: $t_\infty=M$\\
Température instantannée: \(T(t)=T_0e^{-kt}+M(1-e^{-kt})\)\\

\subsubsection*{Croissance logistique}
$M$ = population maximale du milieu\\
$P$ = population\\
$v$ = variation de la population\\
Équation type: \(\frac{dv}{dt}+kMv=k \;\mid v(0)=v_0\)
\begin{gather*}
 y=v \qquad    P(x)=km>0 \qquad q(x)=k
\end{gather*}
Variation limite: \(v\infty=\frac{q}{p}=\frac{k}{kM}=\frac{1}{M}\)\\
Variation instantanée: \(P(t)=\frac{P_0 M}{M e^{-kMt}+P_0 (1-e^{- k M t})}\)\\

\subsection*{Méthode générale}
\subsubsection*{Problème type}
\[\frac{dy}{dt}+Py=Q \;\mid y(0)=y_0, \;P \in\; ] 0,\infty[\;,\; Q \in\; ]-\infty,\infty[\]
É.D linéaire et séparable\\
Facteur intégrant: \(\mu(t)=e^{\int P\:dt}\)\\
Condition initiale: \(y(0)=y_0\)\\
\(y_0=\frac{q}{p}+Ce^{-pt}\)\\
\(C=y_0-\frac{q}{p}\)\\
Solution générale: \(y(t)=\frac{\int e^{pt} Q \:dt+C}{e^{pt}}\)\\
\(y(t)=y_0e^{-pt}+\frac{q}{p}e^{-pt}\)\\
Comportement à l'infini: \(y_\infty(t)=\frac{q}{p}\)\\
donc: \(y(t)=y_0e^{-pt}+y_\infty(1-e^{-pt})\)


